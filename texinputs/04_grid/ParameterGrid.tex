\subsection{Parameter scans on masses, couplings and mixing angles}

\paragraph{Logic of how we proceeded}

\begin{itemize}
\item Starting from benchmark 3 of \cite{Bauer:2017ota}
\item Mapping the kinematics and sensitivity of the model by scanning some of the
various parameters
\item Checking whether other existing models can be rescaled
\end{itemize}

\subsubsection{Results of studies}

Each of the signatures should have the following plots in the planes
of the final recommendation: 
\begin{itemize} 
\item efficiency at parton level with simplified, published cuts
\item total and fiducial cross-section at parton level 
\item 2 - 3 kinematic plots of what has been scanned that are most representative for the analysis (here the analysers decide, then we harmonize at the end)
\end{itemize} 

Signatures:

\begin{itemize}

\item{Mono-Z (lep/had)}

\item{MonoH$\rightarrow$bb}

\item{Monojet}

\item{ttbar+MET}, with specific discussion about rescaling

\item{other signatures who have not yet presented at public meetings, in ATLAS and CMS}

\end{itemize}

\subsubsection{Final proposal for parameter scan}

\begin{itemize}

\item a two-dimensional scan in the light pseudoscalar mass (ma) -
heavy pseudoscalar mass (mA) plane where ma = mA, fixing tanBeta to 1.0,
sinTheta to 0.35 and the Dark Matter mass (mDM) to 10 GeV.


\item a one-dimensional scan in DM mass from 1 GeV to 500 GeV for a
point in the middle of the sensitivity range for the mono-V analyses at
mA=600, ma=250 GeV, so the connection between this model and cosmology
is clear as the measured relic density starts being satisfied at values
of DM mass around 100 GeV

\end{itemize}


In order to explore changes in complementarity with different
analyses and kinematics, this should be complemented by:

\begin{itemize}

\item a two-dimensional scan in the ma − tanBeta plane, for
comparison with the ttbar+MET / bbar+MET analyses. In this case, the
charged Higgs mass (mH+/-), the heavy pseudoscalar mass (mA) and the
heavy Higgs mass (mH) should be fixed to 600 GeV. This scan includes points: 
50, 45, 40, 35, 30, 25, 20, 15, 10, 5
for M(a) masses between 10 and 350 GeV. The high-tanBeta points would be
of primary interest to the HF + DM searches. Uli's studies have shown
that one can simply reweight the existing tt+DM/bb+DM models from DMF to
the new 2HDM+PS cross sections; full simulation of the newly proposed
2HDM+PS points is not required.

\item two one-dimensional scans in $sin_{\theta}$ for the comparison of
mono-Higgs and bbar+MET analysis (it is expected that the bbar+MET
analysis will only have to rescale previous models/cross-sections)
{[}2{]}: - mH± = mA = mH = 600GeV , ma = 200GeV, tanBeta=1 - mH± = mA =
mH = 1000GeV , ma = 350GeV, tanBeta=1

\end{itemize}

The PDF recommended is 5-flavor. ATLAS will use the NNPDF3.0
PDF set. Some text by Fabio Maltoni and Ulrich Haisch can be found in the texinputs\_app folder.  

